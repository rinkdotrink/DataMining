\documentclass[a4paper,12pt,oneside,bibtotoc,numbers=noenddot]{scrreprt}

%Pakete
\usepackage[latin9]{inputenc}
\usepackage[ngerman]{babel}
\usepackage{listings}
\usepackage{graphicx}
\usepackage{BachelorThesis}

% Allgemeine Informationen
\newcommand\mytitle{Titel der Arbeit}
\newcommand\myauthor{Name des Autors oder der Autoren}
\newcommand\mydepartment{Informatik und Elektrotechnik}
\newcommand\myinstitute{Hochschule Zittau/G\"{o}rlitz}
\newcommand\mytutor{Name und Titel des betreuenden Professors}
\newcommand\mySecondTutor{Name und Titel des betrieblichen Betreuers}

% Abstracts
\newcommand\mysubject{Das deutsche Abstract.}
\newcommand\mysubjectenglish{The english abstract.}

% PDF-Einstellungen
\hypersetup
{
	pdftitle = \mytitle,
	pdfsubject = \mysubject,
	pdfauthor = \myauthor,
	pdfkeywords = {},
	colorlinks = {true},
	pdfborder = 0 0 0
}

\begin{document}
\nocite{*}

%
\pagenumbering{alph}
\begin{titlepage}
\thispagestyle{empty} 
 \begin{center}
 \vspace{2.0cm} 
 {\bfseries \huge Verse des Alten Testaments clustern\\}
 \vspace{3.0cm} 
 {\bfseries \huge Belegarbeit\\}
 \vspace{3.0cm}
 {\normalsize eingereicht am Fachbereich\\}
 {\bfseries \Large Informatik\\}
 {\normalsize der Hochschule Zittau/G�rlitz (HAW)\\}
 \vspace{1cm}
 {\normalsize als Pr�fungsleistung im Fach\\}
 {\bfseries \Large Data Mining\\}
 \vspace{1cm}
 {\normalsize vorgelegt von:\\}
 {\bfseries \Large Christof Ochmann (35989)\\
 Ingo K�rner (40586)\\}
 \vspace{1cm}
 {\normalsize  G�rlitz, 11. Juli 2012\\}
 \vspace{0.5cm}
 Betreuer:	Prof. ten Hagen\\
 \vfill
\end{center}
\end{titlepage}

%
%% Kurzreferat
\thispagestyle{empty}
\section*{Abstract}\label{Abstract}
In diesem Projekt werden die Verse des Alten Testaments geclustert. Einander �hnliche Verse sollen durch Clusteralgorithmen in das selben Cluster gruppiert werden. Es wird untersucht, bei wie vielen Attributen und wie vielen Clustern die besten Resultate erzielt werden. Der Clusteralgorithmus soll dabei auf einem handels�blichen Laptop ausgef�hrt werden. Verteiltes Clustern ist nicht Gegenstand dieser Arbeit.


%\mysubject
%\section*{Abstract}
%\mysubjectenglish

\pagenumbering{Roman}
\tableofcontents
\listoffigures
\lstlistoflistings

\begin{listofacronyms}
\acronym{DBMS}{Datenbankmanagementsystem}
\acronym{ERD}{Entity-Relationship Diagram}
\acronym{OLAP}{Online Analytical Processing}
\acronym{SQL}{Structured Query Language}

\end{listofacronyms}

\begin{flushleft}
\begin{thebibliography}{sotief}
\bibitem{bib1}{Martin, Robert C. (2008): Clean Code: A Handbook of Agile Software Craftsmanship. Prentice Hall International}

\bibitem{bib2}{Freeman, Eric (2007): Entwurfsmuster von Kopf bis Fu�. O'REILLY}

\bibitem{bib3}{Peter Eisentraut, Bernd Helmle. (2009): PostgreSQL-Administration. O'REILLY}

\bibitem{bib4}{\begin{verbatim}http://www.postgresql.org/ (08.06.2012)\end{verbatim}} 

\bibitem{bib5}{\begin{verbatim}http://wiki.postgresql.org/ (08.06.2012)\end{verbatim}} 



\end{thebibliography}
\end{flushleft}

\newpage
\pagestyle{chapterStyle}
\pagenumbering{arabic}

\chapter{Theorie}
\section{Einleitung}\label{Einleitung}

\section{Aufgabenstellung}\label{Aufgabenstellung}
Durch Clustering werden �hnlichkeiten in gro�en Datenbest�nden gefunden. In diesem Projekt werden mit Hilfe von Clustering-Algorithmen Verse des AT geclustert. Dabei sollen einander �hnliche Verse in einem Clustern zusammenfasst werden, d.h. Datens�tze, die sich �hneln, kommen in dasselbe Cluster. In diesem Projekt werden verschiedene Clusteralgorithmen auf das Alte Testament angewendet. In Werkzeugen wie Weka oder ELKI sind diese Algorithmen bereits implementiert und k�nnen genutzt werden.
Es soll nur auf einer Maschine geclustert werden, d.h. skalierbares Datamining mit Apache Mahout wird in dieser Arbeit nicht behandelt.
\section{Relevanz des Forschungsgegenstandes}\label{RelevanzDesForschungsgegenstandes}

\section{Der aktuelle Wissensstand}\label{DerAktuelleWissensstand}


\chapter{Umsetzung}\label{Umsetzung}

\section{Zusammenfassung}\label{Zusammenfassung}

\section{Ausblick}\label{Ausblick}
In einem Folgeprojekt k�nnte mit Werkzeugen wie Apache Mahout verwendet werden, mit dem auch verteilt geclustert werden kann.



%\chapter{Theoretische Grundlagen}
%Die f\"{u}r den Untersuchungsgegenstand relevanten Themen, die \"{u}ber die
%grundlegenden Studieninhalte hinausgehen; oft auch anwendungsspezifische Aspekte - %ca. 6 Seiten

%\chapter{Ist-Analyse}
%Welche Defizite sollen mit der Arbeit behoben werden, welche nicht? %Pr\"{a}zisierung
%der Zielstellung - ca. 6 Seiten

%\chapter{L\"{o}sungskonzept}
%Wie sollen die Defizite behoben werden? Methoden, fachliche Auseinandersetzung
%mit alternativen Ans\"{a}tzen und Auffassungen, Systembeschreibung (Architektur,
%Vorgehensmodell, \ldots) - ca. 12 Seiten

%\chapter{Implementierung}
%Umsetzung des L\"{o}sungskonzepts, Begr\"{u}ndung der verwendeten Technologien - %ca. 8
%Seiten

%\chapter{Ergebnisse}
%Objektive Bewertung der vorliegenden L\"{o}sung, diverse Testverfahren,
%Nutzerbefragungen - ca. 4 Seiten

%\chapter{Fazit und Ausblick}
%Zusammenfassung s\"{a}mtlicher Ergebnisse in Bezug auf die Zielerf\"{u}llung und
%Vorschl\"{a}ge f\"{u}r weiterf\"{u}hrende Arbeiten - ca. 2 Seiten

\bibliographystyle{alphadin}
\begin{appendix}
\newpage
\pagestyle{appendixAStyle}
\chapter{Codebeispiele}
\begin{lstlisting}[caption=alle Tabellen erstellen, firstnumber=1]{code:TabellenErstellen}
CREATE TABLE "user"
(
  userid bigint,
  name text,
  email text,
  gender text,
  birthday date,
  password text,
  image text
)
WITH (
  OIDS=FALSE
);
ALTER TABLE "user"
  OWNER TO postgres;

CREATE TABLE event
(
  eventid bigint NOT NULL,
  creatorid bigint,
  date date,
  eventname text,
  occasion text,
  location text,
  lon double precision,
  lat double precision,
  description text,
  numbermaleconfirmed int,
  numberfemaleconfirmed int
)
WITH (
  OIDS=FALSE
);
ALTER TABLE event
  OWNER TO postgres;

CREATE TABLE message
(
  messageid bigint,
  eventid bigint,
  senderid bigint,
  recipientid bigint,
  textmessage text,
  date date
)
WITH (
  OIDS=FALSE
);
ALTER TABLE message
  OWNER TO postgres;

CREATE TABLE participation
(
  participationid bigint,
  userid bigint,
  eventid bigint
)
WITH (
  OIDS=FALSE
);
ALTER TABLE participation
  OWNER TO postgres;
\end{lstlisting}

\begin{lstlisting}[caption=Datenimport �ber COPY, firstnumber=1]{code:COPY}
COPY public.Event (eventid, creatorid, date, eventname, occasion, location, lon, lat, description, numbermaleconfirmed, numberfemaleconfirmed) From 'C:\Event.txt' DELIMITER ';';
COPY public.Message (messageid, eventid, senderid, recipientid, textmessage, date) From 'C:\Message.txt' DELIMITER ';';
COPY public.Participation (participationid, userid, eventid) From 'C:\Participation.txt' DELIMITER ';';
COPY public.User (userId, name, email, gender, birthday, password, image) From 'C:\User.txt' DELIMITER ';';
\end{lstlisting}


\begin{lstlisting}[caption=Prim�r- und Fremdschl�ssel hinzuf�gen, firstnumber=1]{code:PrimaryForeignKeys}
ALTER TABLE public.event ADD PRIMARY KEY (eventid);
ALTER TABLE public.message ADD PRIMARY KEY (messageid);
ALTER TABLE public.participation ADD PRIMARY KEY (participationid);
ALTER TABLE public.user ADD PRIMARY KEY (userid);

ALTER TABLE event ADD CONSTRAINT event_creatorid FOREIGN KEY (creatorid) REFERENCES public.user (userid) MATCH FULL;
ALTER TABLE message ADD CONSTRAINT message_eventid FOREIGN KEY (eventid) REFERENCES event (eventid) MATCH FULL;
ALTER TABLE message ADD CONSTRAINT message_senderid FOREIGN KEY (senderid) REFERENCES public.user (userid) MATCH FULL;
ALTER TABLE message ADD CONSTRAINT message_recipientid FOREIGN KEY (recipientid) REFERENCES public.user (userid) MATCH FULL;
ALTER TABLE participation ADD CONSTRAINT participation_userid FOREIGN KEY (userid) REFERENCES public.user (userid) MATCH FULL;
ALTER TABLE participation ADD CONSTRAINT participation_eventid FOREIGN KEY (eventid) REFERENCES event (eventid) MATCH FULL;
\end{lstlisting}


\begin{lstlisting}[caption=Indexe auf Spalten legen, firstnumber=1]{code:Indexe}
CREATE INDEX event_creatorid ON public.event(creatorid);
CREATE INDEX message_eventid ON public.message(eventid);
CREATE INDEX message_senderid ON public.message(senderid);
CREATE INDEX message_recipientid ON public.message(recipientid);
CREATE INDEX participation_userid ON public.participation(userid);
CREATE INDEX participation_eventid ON public.participation(eventid);

CREATE INDEX event_date ON public.event(date);
CREATE INDEX event_eventname ON public.event(eventname);
CREATE INDEX event_occasion ON public.event(occasion);
CREATE INDEX event_location ON public.event(location);
CREATE INDEX event_lon ON public.event(lon);
CREATE INDEX event_lat ON public.event(lat);
CREATE INDEX event_numbermaleconfirmed ON public.event(numbermaleconfirmed);
CREATE INDEX event_numberfemaleconfirmed ON public.event(numberfemaleconfirmed);

CREATE INDEX message_textmessage ON public.message(textmessage);
CREATE INDEX message_date ON public.message(date);

CREATE INDEX user_name ON public.user(name);
CREATE INDEX user_email ON public.user(email);
CREATE INDEX user_gender ON public.user(gender);
CREATE INDEX user_birthday ON public.user(birthday);
\end{lstlisting}
\end{appendix}

\newpage
\chapter{Arbeitsaufteilung}


\begin{table}[h] \begin{flushleft}  \begin{tabular}{|l||c|c|c|c|c|c|}
\hline
\textbf{Arbeit}		&	\textbf{C. Ochmann}	& \textbf{I. K�rner}  \\ \hline \hline
Abstract   	      &                     & 0       \\
Einleitung  &                             		      & ~\ref{Einleitung} \\
Aufgabenstellung&                                  & ~\ref{Aufgabenstellung}  \\
Forschungsgegenstand&                              & ~\ref{RelevanzDesForschungsgegenstandes} \\ 
akt. Wissensstand&                                      & ~\ref{DerAktuelleWissensstand}  \\ 
Testrechner&                                      & ~\ref{Testrechner}  \\ 
Der Aufbau des AT&                                      & ~\ref{AufbauAT}  \\ 
ARFF&                                      & ~\ref{ARFF}  \\ 
Clustering-Verfahren&       ~\ref{Clustering}                                &  \\ 
Partitionierende Verfahren&       ~\ref{Partitionierende}                                &  \\ 
Austauschverfahren mit Zielfunktion&       ~\ref{Austausch}                        &  \\ 
Minimaldistanz-Verfahren&       ~\ref{Minimal}                        &  \\ 
Minimaldistanz-Verfahren&       ~\ref{Minimaldistanz}                        &  \\ 
k-Means-Verfahren&       ~\ref{kmeans}                        &  \\ 
EM-Algorithmus&       ~\ref{em}                        &  \\ 
DBSCAN&       ~\ref{dbscan}                        &  \\ 
Hierarchische-Verfahren&       ~\ref{Hierarchische}                        &  \\ 
Agglomerative-Verfahren&       ~\ref{Agglomerative}                        &  \\ 
Wie AT in das ARFF-Format �berf�hren?&                                      & ~\ref{ATzuARFF}  \\ 
Analyse Text2ARFFConverter&                                      & ~\ref{AnalyseText2ARFFConverter}  \\ 
Entwurf Text2ARFFConverter&                                      & ~\ref{EntwurfText2ARFFConverter}  \\ 
Nach welchen W�rtern clustern?&                                      & ~\ref{WelcheWoerter}  \\ 
H�rden beim Einlesen der ARFF-Datei&                                      & ~\ref{Huerden}  \\ 
SimpleKMeans&                                      & ~\ref{SimpleKMeans}  \\ 
Weitere Cluster-Algorithmen&                                      & ~\ref{weitereClusterAlgorithmen}  \\ 
Instanzen in Verse verwandeln&                                      & ~\ref{Text2ClusterFile}  \\ 
AT mit SimpleKMeans geclustert&                                      & ~\ref{SimpleKMeansgeclustert}  \\ 
Zusammenfassung&         																& ~\ref{Zusammenfassung} \\ 
Ausblick&        																					 & ~\ref{Ausblick} \\ 
\hline \hline
\end{tabular} \end{flushleft} \caption{Aufteilung} \end{table}



\newpage
\chapter{Eigenst�ndigkeitserkl�rung}
Hiermit erkl�re ich, dass ich diese Arbeit selbst�ndig verfasst habe. Mir ist bekannt, dass jede Form des Plagiats mit der Note 5 (Betrugsversuch) bewertet wird.

\begin{tabular}{@{}p{6.0cm}p{6.0cm}}	  		  		 	
	  		 & \\
	  		 & \\
				  			\textbf{Ochmann, Christof} &               Unterschrift:\\				 
				 &\\
				 & \\
				  			\textbf{K�rner, Ingo}   	&                     Unterschrift:\\			
\end{tabular}
\end{document}
