\section{Relevanz des Forschungsgegenstandes}\label{RelevanzDesForschungsgegenstandes}
Der Forschungsgegenstand dieser Arbeit ist, die Verse des AT mit verschiedenen Clusteralgorithmen auf einem handels�blichen Laptop zu clustern. Daf�r wird aus einer Menge m�glicher Clustersoftware eine passende ausgew�hlt. Der Forschungsgegenstand ist relevant, da bisher noch keine Ergebnisse f�r das Clustern von Versen des AT mit dem Testrechner vorliegen. Ziel der Forschung ist es, geeignete Clusteralgorithmen zu finden und mit verschiedenen Parametern auszuf�hren - wie Anzahl Cluster bzw. Anzahl Attribute nach denen geclustert werden soll.
Es wird sich vertiefend in eine Clustersoftware und den Clusteralgorithmen eingearbeitet. Das geschieht z.B. unter Zuhilfenahme von B�chern und Online-Ressourcen. In diesen Medien ist der Forschungsstand zu Software und Algorithmen dokumentiert. Bei der Erstellung der passenden Eingabeformate und zur Auswertung der geclusterten Ergebnisse sind zudem Programme zu erstellen, bei denen technische Probleme gel�st werden m�ssen.