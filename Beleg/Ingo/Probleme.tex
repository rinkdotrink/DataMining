\section{H�rden beim Einlesen der ARFF-Datei}\label{Huerden}
Die erzeugte ARFF-Datei wird f�r das AT 900 MB gro�. In Weka kommt auf dem Testrechner beim Einlesen der 900MB gro�en ARFF-Datei die Fehlermeldung "`OutOfMemory"'. Und Weka hat sich darauf hin geschlossen.
In der Konfigurationsdatei RunWeka.ini konnte der Parameter maxheap nur auf h�chstens 1550m gesetzt werden. Wird mehr Platz f�r den heap space angegeben, kommt eine Fehlermeldung von der JVM.
Darauf hin wurde von einem 32Bit-Java auf ein 64Bit-Java gewechselt. Damit verschwanden beide Fehlermeldungen und es konnte der maxheap auf 8000m gesetzt werden.

Wird die ARFF-Datei eingelesen, braucht Weka bzw. die JVM daf�r 6,8 GB Arbeitsspeicher. Auf dem Entwicklungsrechner stehen aber nur 8GB zur Verf�gung. Es wird davon ausgegangen, dass f�r die Anwendung einiger Clusteralgorithmen der restliche Arbeitsspeicher nicht ausreicht und die Geschwindigkeit durch Swapping ausgebremst wird.

Um die Gr��e der ARFF-Datei zu reduzieren, gibt es das Format Sparse ARFF, dass auch gro�e Datenmengen kompakt speichern kann.

Sparse ARFF Dateien sind gew�hnlichen AARF Dateien �hnlich, au�er das Attribute mit dem Wert 0 nicht repr�sentiert werden. Nicht-Null Attribute werden durch die Attributnummer und den Wert angegeben.

Durch Sparse ARFF konnte die Dateigr��e von 900MB auf 4 MB reduziert werden.
Die Datei wird dadurch auch schneller von Weka eingelesen und Weka braucht weniger RAM.