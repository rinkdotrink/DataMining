\section{ARFF}\label{ARFF}
Damit die Clusteralgorithmen auf das AT angewandt werden k�nnen, muss das AT vorher in das ARFF-Format \cite{bib3} (Attribute-Relation File Format) umgewandelt werden.

ARFF (Attribute-Relation File Format) ist eine Textdatei, die eine Liste von Instanzen beschreibt, die sich eine Menge von Attributen teilen. Eine Instanz w�re in unserem Beispiel ein Vers des AT. Die Attribute, die sich ein Vers mit anderen Versen teilt, sind die W�rter aus denen der Vers besteht.
Der Vers "`1. Am Anfang schuf Gott Himmel und Erde."' besteht aus der Attribut-Menge \{1, Am, Anfang, schuf, Gott, und, Erde, Himmel\}

Die Elemente "`1"', "`Himmel"', "`und"', "`Erde"' teilt sich der Vers z.B. mit dem Vers "`1. Also ward vollendet Himmel und Erde mit ihrem ganzen Heer."'

Man kann sagen, dass diese zwei Instanzen bzw. Verse deswegen eine gewissen �hnlichkeit haben, da sie sich einige gemeinsame Attribute teilen.

Eine ARFF-Datei besteht aus zwei Teilen. Dem Header- und dem Data-Teil. Im Header-Teil befinden sich die Attribute. Zu beachten ist, dass es sich bei dem Wert eines Attributes um die H�ufigkeit handelt, wie oft das Attribut im Vers vorkommt. Jedes Attribute hat einen Datentyp. Im vorliegenden Fall handelt es sich um Zahlen.
Im Datenteil befinden sich die Instanzen, d.h. die Verse. Sie bestehen aus den H�ufigkeiten, die mit Komma getrennt sind. Eine Instanz im Datenteil hat so viele mit Komma getrennte Zahlen, wie es Attribute im Header gibt. Jede Zahl steht f�r ein Attribut im Header. Die Position eines Attributes im Header stimmt mit der Position des Attributes in der Instanz �berein. D.h. das die dritte Zahl einer Instanz f�r das Attribut "`Anfang"' steht, da dieses auch an dritter Position im Header auftaucht.

Im folgenden sind drei Verse gegeben:

1. Buch Mose \\
Kapitel 1 \\
1. Am Anfang schuf Gott Himmel und Erde. \\

F�r diese drei Verse werden zehn Attribute definiert:

\% 1. Title: AT
@RELATION AT \\

@ATTRIBUTE 1 NUMERIC \\
@ATTRIBUTE Am NUMERIC \\
@ATTRIBUTE Anfang NUMERIC \\
@ATTRIBUTE schuf NUMERIC \\
@ATTRIBUTE Gott NUMERIC \\
@ATTRIBUTE Himmel NUMERIC \\
@ATTRIBUTE und NUMERIC \\
@ATTRIBUTE Erde NUMERIC \\
@ATTRIBUTE Fluss NUMERIC \\
@ATTRIBUTE Feld NUMERIC \\

Im Daten-Teil wird jeder Vers durch eine Instanz ausgedr�ckt. Die Instanz hat soviele Zahlen, wie es Attribute im Header-Teil gibt. Eine 0 bedeutet, das Attribut kommt im Vers nicht vor. Eine 1 bedeutet, das Attribut kommt im Vers einmal vor.

@DATA \\
1,0,0,0,0,0,0,0,0,0 \\
1,0,0,0,0,0,0,0,0,0 \\
1,1,1,1,1,1,1,1,0,0 \\

