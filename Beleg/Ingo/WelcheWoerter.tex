\section{Nach welchen W�rtern sollte geclustert werden?}\label{WelcheWoerter}
Bei Versuchen hat sich gezeigt, dass die Cluster ausgewogener gef�llt sind, wenn die h�ufigsten W�rter gew�hlt werden. Werden die seltensten W�rter gew�hlt, entstehen viele leere Instanzen und es sammeln sich alle Verse in einigen wenigen Clustern und die restlichen Cluster sind fast leer. Wahrscheinlich ist das damit zu begr�nden, dass beim Clustern �hnliche Verse in ein Cluster kommen. Wenn nun aber durch die seltensten W�rter die Verse alle sehr unterschiedlich sind, wird es f�r den Algorithmus schwierig, �hnlichkeiten zu finden. Mit einem Wort, dass nur einmal vorkommt, kann man einen Vers gut von anderen Versen unterscheiden, aber es wird nicht helfen, zu sehen, wo die �hnlichkeiten zu anderen Versen sind. Wenn dagegen W�rter gew�hlt werden, die h�ufig vorkommen, entstehen Cluster, die ausgewogender gef�llt sind. Es kommt mit h�herer Wahrscheinlichkeit zwei Verse in dasselbe Cluster, wenn sie an den selben Positionen die gleichen H�ufigkeiten haben.