\newpage
\chapter{Arbeitsaufteilung}


\begin{table}[h] \begin{flushleft}  \begin{tabular}{|l||c|c|c|c|c|c|}
\hline
\textbf{Arbeit}		&	\textbf{C. Ochmann}	& \textbf{I. K�rner}  \\ \hline \hline
Abstract   	      &                     & 0       \\
Einleitung  &                             		      & ~\ref{Einleitung} \\
Aufgabenstellung&                                  & ~\ref{Aufgabenstellung}  \\
Forschungsgegenstand&                              & ~\ref{RelevanzDesForschungsgegenstandes} \\ 
akt. Wissensstand&                                      & ~\ref{DerAktuelleWissensstand}  \\ 
Testrechner&                                      & ~\ref{Testrechner}  \\ 
Der Aufbau des AT&                                      & ~\ref{AufbauAT}  \\ 
ARFF&                                      & ~\ref{ARFF}  \\ 
Clustering-Verfahren&       ~\ref{Clustering}                                &  \\ 
Partitionierende Verfahren&       ~\ref{Partitionierende}                                &  \\ 
Austauschverfahren mit Zielfunktion&       ~\ref{Austausch}                        &  \\ 
Minimaldistanz-Verfahren&       ~\ref{Minimal}                        &  \\ 
Minimaldistanz-Verfahren&       ~\ref{Minimaldistanz}                        &  \\ 
k-Means-Verfahren&       ~\ref{kmeans}                        &  \\ 
EM-Algorithmus&       ~\ref{em}                        &  \\ 
DBSCAN&       ~\ref{dbscan}                        &  \\ 
Hierarchische-Verfahren&       ~\ref{Hierarchische}                        &  \\ 
Agglomerative-Verfahren&       ~\ref{Agglomerative}                        &  \\ 
Wie AT in das ARFF-Format �berf�hren?&                                      & ~\ref{ATzuARFF}  \\ 
Analyse Text2ARFFConverter&                                      & ~\ref{AnalyseText2ARFFConverter}  \\ 
Entwurf Text2ARFFConverter&                                      & ~\ref{EntwurfText2ARFFConverter}  \\ 
Nach welchen W�rtern clustern?&                                      & ~\ref{WelcheWoerter}  \\ 
H�rden beim Einlesen der ARFF-Datei&                                      & ~\ref{Huerden}  \\ 
SimpleKMeans&                                      & ~\ref{SimpleKMeans}  \\ 
Weitere Cluster-Algorithmen&                                      & ~\ref{weitereClusterAlgorithmen}  \\ 
Instanzen in Verse verwandeln&                                      & ~\ref{Text2ClusterFile}  \\ 
AT mit SimpleKMeans geclustert&                                      & ~\ref{SimpleKMeansgeclustert}  \\ 
Zusammenfassung&         																& ~\ref{Zusammenfassung} \\ 
Ausblick&        																					 & ~\ref{Ausblick} \\ 
\hline \hline
\end{tabular} \end{flushleft} \caption{Aufteilung} \end{table}

